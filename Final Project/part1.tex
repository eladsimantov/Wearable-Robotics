\documentclass[10pt]{article}
	
\usepackage[margin=1in]{geometry}		% For setting margins
\usepackage{amsmath}				% For Math
\usepackage{fancyhdr}				% For fancy header/footer
\usepackage{graphicx}				% For including figure/image
\usepackage{float}
\usepackage{hyperref}
\usepackage{tabto}
\usepackage{cancel}					% To use the slash to cancel out stuff in work
% avoid all eq numbering via:
\usepackage{mathtools}
\mathtoolsset{showonlyrefs}

\usepackage{cite}
\usepackage{url}
\usepackage{amssymb}                % To include mathbb symbols
\usepackage{graphicx}               % In preamble
\newcommand{\mcG}{\mathcal{G}}
\newcommand{\mcU}{\mathcal{U}}
\newcommand{\mcV}{\mathcal{V}}
\hypersetup{colorlinks=true,
            urlcolor=blue}
            
% Set up fancy header/footer
\pagestyle{fancy}
\fancyhead[LO,L]{Wearable Robotics 0360108}
\fancyhead[CO,C]{Final Project - Part 1}
\fancyhead[RO,R]{\today}
\fancyfoot[LO,L]{}
\fancyfoot[CO,C]{\thepage}
\fancyfoot[RO,R]{}
\renewcommand{\headrulewidth}{0.1pt}
\renewcommand{\footrulewidth}{0.1pt}
% \renewcommand{\thesubsection}{(\roman{subsection})}
\renewcommand{\thesubsubsection}{(\roman{subsubsection})}
\usepackage{algorithm}
\usepackage{algorithmic}


%%%%%%%%%%%%%%%%%%%%%%

\begin{document}
\begin{table}[h]
    \centering
    \begin{tabular}{l l l}
        \hline
        Name & ID & Email \\
        \hline
        Elad Siman Tov & - & elad.sim@campus.technion.ac.il \\
        \hline
        Eitan Gerber & - & eitangerber@campus.technion.ac.il \\
        \hline
    \end{tabular}
    \label{tab:personal_info}
\end{table}
\noindent Our code (.ino file) is \href{https://github.com/eladsimantov/Wearable-Robotics/blob/main/Final%20Project/Potentiometer_Servos.ino}{published on Github} for reference.
A video of the hand successfully opening and closing based on the potentiometer \href{https://technionmail-my.sharepoint.com/:v:/g/personal/eitangerber_campus_technion_ac_il/Ead1dbEn8xNEuNcpJJbua7sBZmVPV9wEbOwiU09-S7oXpg?e=Y54jJX}{can be found here}.

% -------------------------------------- %
% --------- Assembly -------------- %
% -------------------------------------- %
\section*{Assembly}
During the assembly process we encountered some difficulties in ...

\section*{Operation}
To demonstrate the operation of the hand we have programmed the micro-controller to map the potentiometer signal range (1024 bits) to the appropriate range of motion we have predefined for each servo motor corresponding to each finger or the wrist. 
The wrist servo motor was reset at its mid range position to allow for both supination and pronation actions. The rest of the servo motors (fingers) were reset to their zero position to avoid slacking the string driving each finger.
The video demonstrates the successful opening and closing of the hand. 

\end{document}